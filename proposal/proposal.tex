%%%%%%%%%%%%%%%%%%%%%%%%%%%%%%%%%%%%%%%%%%%%%%%%%%%%%%%%%%%%%%%%%%%%%%%%%%%%%%%%
%2345678901234567890123456789012345678901234567890123456789012345678901234567890
%        1         2         3         4         5         6         7         8

\documentclass[letterpaper, 10 pt,noend]{article}  % Comment this line out if you need a4paper

%\documentclass[a4paper, 10pt, conference]{ieeeconf}      % Use this line for a4 paper

%\IEEEoverridecommandlockouts                              % This command is only needed if 
                                                          % you want to use the \thanks command

%\overrideIEEEmargins                                      % Needed to meet printer requirements.

% See the \addtolength command later in the file to balance the column lengths
% on the last page of the document

% numbers option provides compact numerical references in the text. 
\usepackage[font=small,labelfont=bf]{caption}

% The following packages can be found on http:\\www.ctan.org
\usepackage{graphics} % for pdf, bitmapped graphics files
\usepackage{epsfig} % for postscript graphics files
\usepackage{amsmath} % assumes amsmath package installed
\usepackage{amssymb}  % assumes amsmath package installed
\usepackage{listings}
\usepackage{graphicx,float,wrapfig}
\usepackage{subfig}
\usepackage{algpseudocode}
\usepackage{algorithm}
\usepackage{mathtools}
\usepackage{cite}
\usepackage{color}
\usepackage[usenames,dvipsnames,svgnames,table]{xcolor}

%% Define symbols and special formats for texts
\DeclareMathOperator{\F}{\rotatebox[origin=c]{45}{$\Box$}}
\DeclareMathOperator{\G}{\Box}
\DeclareMathOperator{\X}{\bigcirc}
\DeclareMathOperator{\Cox}{\scalebox{0.8}{$\F$}\hspace{-9.3pt}\bigcirc}
\newcommand{\NN}{\mathbb{N}}
\newcommand{\FALSE}{\mathbf{false}}
\newcommand{\TRUE}{\mathbf{true}}

\title{\LARGE \bf CS 6751 Intro to Mobile Robot Manipulation Project Proposal}


\author{Scott Hamill, Gangyuan Jing, Kai Weng Wong% <-this % stops a space
}


\begin{document}


\maketitle
\thispagestyle{empty}
\pagestyle{empty}

% add our sections here
\section{Introduction and Background}


Creating robot controllers by combining high-level task planning with low level motion planning has been a topic of recent interest \cite{HKG2009,Belta2008,Bhatia2011,Dornhege2009,Erdem2011,Tom2014,Srivastava2014}.
The integration of discrete task reasoning and motion planning allows users to specify complex robot behaviors and tasks from a high-level perspective while satisfying low level motion constraints.

Past research has shown promising results on controller synthesis with formal methods \cite{HKG2009,Belta2008,Bhatia2011,Wongpiromsarn,Maly,Vasile}.
Recent research has addressed controller synthesis using formal methods, specifying tasks and behaviors in a formal language such as Linear Temporal Logic (LTL) \cite{HKG2009,Belta2008,Bhatia2011,Dornhege2009,Erdem2011,Tom2014,Srivastava2014}.
Controllers synthesized in this manner are correct-by-construction, thereby providing guarantees on the behavior of the robot.
These algorithms typically abstract the robot system and environment by discretizing the workspace.
This technique works well for mobile robot systems, but discretizations for manipulation systems are typically intractable due to combinatorial blow-up of the state space.

Formal methods have previously been applied to controller synthesis for manipulation \cite{Dornhege2009,Erdem2011,Tom2014,Srivastava2014,CambonAG09,KaelblingL11,PlakuH10,HeLKV15}.
Most of these works utilize AI based planners for high-level task planning and low-level motion planners for finding feasible robot trajectories \cite{Nau,Hoffmann01}.
While these frameworks allow users to specify high-level behaviors and task specifications, the resulting behaviors are non-reactive in that the robot behavior does not depend on the state of the environment.

This work addresses synthesizing correct-by-construction controllers for a manipulation task from a high-level behavioral specification.
This framework provides three main contributions: a high-level {\it task planner} for determining sequences of high-level actions, a {\it trajectory planner} for finding and executing feasible manipulator motions, and an {\it action planner} for performing precise manipulator motion and actuation.
This framework is is demonstrated on a Baxter robot (Rethink Robotics) assembling a series of modular robot components into a pre-defined configuration. 


\begin{comment}
With the development of hardware and software of robots,
There is an increasing interest in combining high-level task planning
with low-level motion planning when creating robot controllers \cite{HKG2009,Belta2008,Bhatia2011,Dornhege2009,Erdem2011,Tom2014,Srivastava2014}.
While integrating tasking and motion planning enables users to specify complex robot tasks from a high-level perspective,
combining discrete task reasoning with continuous motion planning brings challenges to designing robot controllers.

Past research has shown promising results on controller synthesis with formal methods \cite{HKG2009,Belta2008,Bhatia2011,Wongpiromsarn,Maly,Vasile}.
In these papers, robot tasks are expressed in formal language such as Linear Temporal Logic (LTL).
Controllers can be automatically synthesized from the task specification,
while providing guarantees on the correctness of robot behaviours.
These controller synthesis algorithms typically abstract robot systems and environment by discretizing the robot workspace,
in order to avoid the combinatorial blow up of the state space.
While the abstraction reduces the complexity of mobile systems,
the discretization is usually intractable for manipulation systems,
due to high degrees of freedom of manipulation tasks.

Many frameworks were developed on controller synthesis with formal methods for manipulation \cite{Dornhege2009,Erdem2011,Tom2014,Srivastava2014,CambonAG09,KaelblingL11,PlakuH10,HeLKV15}.
Most of these works utilize AI based planners for high-level task planning and low-level motion planners for finding feasible robot trajectories \cite{Nau,Hoffmann01}.
These frameworks allows user to control manipulation systems with high-level task specifications.
However, high-level planners used in these frameworks limit possible tasks to non-reactive,
i.e. the behavior of the robot does not depend on the state of the environment.

Moreover, formal methods are also used in robotics to generate controllers for coordination of a team of robots \cite{ChenDSB12,KaramanF08,GuoTD14,VasileB14}.
In these work, tasks are specified in Temporal Logic and controllers are automatically synthesized and distributed to each robot agent. 
However, little work has been done on task and motion planning for multi-agent manipulation system with formal method.

In this work, we are interested in developing a framework for automatically synthesizing correct-by-construction controllers
for multi-agent manipulation tasks expressed in Linear Temporal Logic. The framework consists of three main components:
a high-level \textit{task planner} for generating and distributing controllers to each agent,
a \textit{trajectory planner} for finding feasible path and moving each manipulator while avoiding collisions,
and and \textit{action planner} for creating precise motion and actuation for each manipulator.
The framework will be demonstrated with an experiment of two physical Baxter robots assembling a set of modular robots. 
\end{comment}

\section{Task Planning}
The task planner in this work is based on the framework introduced in (\TODO{hadas paper}).
\begin{itemize}
\item Change specification language for multirobot
\item How to distribute tasks from a single automaton
\item How to deal with the situation when the trajectory planner returns not path?
\end{itemize}
\section{Sensing}
How to detect the location of the module
%\bibliographystyle{plain}
%\bibliography{ref}

\end{document}
