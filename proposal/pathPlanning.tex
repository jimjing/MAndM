\section{Path Planning (Catherine)}

This component is divided into two subsections:
\begin{enumerate*}[label=\thesection.\arabic*]
\item Trajectory Planner for robot arm trajectory planning, and
\item Trajectory Executor for robot arm trajectory execution.
\end{enumerate*}

\subsection{Trajectory Planning}

During execution, the task planner requests the path planner to create trajectory plans for one or more robot arms. The planner aims to find a plan that (i) each arm starts at its initial configuration, and (ii) each arm ends with each robot end effector within a certain radius from the final desired workspace position. If the planner finds a plan, it returns the joint-space trajectory for each arm. If the planning exceeds the time limit, then the planner aborts and returns no trajectories. 

There are a variety of path planning algorithms developed in the community~\cite{DBLP:books/daglib/0016830} and some algorithms are adapted for finding robot arm trajectories. 
Researchers have proposed sampling-based approaches with probabilistic completeness such as the Rapidly-Exploring Random Tree (RRT) algorithm ~\cite{VahrenkampBAKD09} to find trajectories for robot arms. Others have used Probabilistic RoadMaps (PRM) that work with high-dimensional spaces~\cite{KavrakiSLO96} but the algorithm requires pre-computation of the roadmap. In this project, we will not conduct pre-processing so we are using RRT to find trajectories for the robot arms.

With RRT, there are still multiple ways to generate a trajectory plan.
For example, if there are four arms available for a task, or two Baxters, then the planning could be synchronous, i.e, we plan all the arms at the same time and each node in RRT stores the joint information of the four arms. With each arm having 7 degrees of freedom (DoF), a synchronous planning has up to 28 degrees of freedom.
With this approach, assuming the workspace has no other obstacles, collision avoidance with the other arms is taken care of during the planning phase so it is unnecessary during trajectory execution.  
The trade off of synchronous planning is that one arm may wait for the others even though its trajectory is found.

Alternatively, another way would be to first create a plan each arm separately and assign priorities to each arm. An arm then replans only when its trajectory intersects with the plan of another arm with a higher priority. Compared with the synchronous approach, each planning contains 7 degrees of freedom but re-plannings of trajectory can go up to (n-1) times, with n being the number of arms. We plan to try both approaches and compare the performance.

There are also path planning libraries such as the Open Motion Planning library (OMPL)~\cite{sucan2012the-open-motion-planning-library} with RRT off the shelf. Compared with  creating our own RRT planner, OMPL has lots of planning algorithms available, but most examples of OMPL work with single robot or arm and planning for multiple arms simultaneously with OMPL may be unfeasible. We plan to learn more OMPL and then decide if we are creating our own RRT planner or we are using OMPL to build our planner.

During the planning phrase, we can also optimize the final trajectory by reducing the difference in joint angles between two nodes in the RRT tree. We use inverse kinematics find robot joint angles given a desired location of the end effector.


%To conduct a path planning of the robot arms, the planner takes in:

%\begin{itemize}
%\item the number of arms we are planning, 
%\item the starting joint configurations of each arm, and
%\item the final workspace positions of robot end effectors.
%\end{itemize}

\subsection{Trajectory Executor}

Once the task planner receives a plan, it can invoke the trajectory executor to execute the path. The executor takes in a joint-space trajectory and notifies the task planner when it finishes the execution. The task planner then invokes proximity planning by the lowest-level controller in the next section to conduct fine and accurate grasping of objects.
