\section{Task Planning}
The task planner in this work is based on the framework introduced in (\TODO{hadas paper}).
A reactive robot task specification is expressed in LTL formulas.
Then the specification is automatically transformed to a correct-by-construction discrete controller.
At last, the controller is continuously implemented to generate desired robot behaviors.
Different from (\TODO{hadas paper}), in this work we aim to synthesize controllers for multiple robots executing a manipulation task.
Therefore, there are three main challenges in this work for task planning, as described in the following sections.

\subsection{Specification Language for Multi-robot}
In (\TODO{cite regex paper, new calin paper}), the author introduces an approach for specifying robot tasks in formal language and generating controllers for a team of robots. 
However, the type of robot tasks is limited to non-reactive, i.e. the environment is assume static and the robot behavior does not depends on the environment state.
The framework in (\TODO{hadas paper}) allows reactive robot task, such as "if the robot sense a soda can, bring the can to the kitchen".
However, the task specification only issues commands for a single robot.
In order to specify reactive tasks for a team of robots, we need to extend the specification language to able to express  multi-robot tasks.

\subsection{Distribute Tasks}
The framework employed in this work will generate a centralized controller for a team of robot.
We will then distribute tasks to each robots in a synchronized process.
One method is to assign tasks to each robot in the task specification and thus the synthesized controller will have
\subsection{Feedback from Trajectory Planner}

