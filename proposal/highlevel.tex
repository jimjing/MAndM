\section{Task Planning (Jim)}
The task planner in this work is based on the framework introduced in \cite{HKG2009}.
A reactive robot task specification is expressed in LTL formulas.
Then the specification is automatically transformed to a correct-by-construction discrete controller.
At last, the controller is continuously implemented to generate desired robot behaviors.
Different from \cite{HKG2009}, in this work we aim to synthesize controllers for multiple robots executing a manipulation task.
Therefore, there are three main challenges in this work for task planning, as described in the following sections.

\subsection{Specification Language for Multi-robot}
In \cite{ChenDSB12,Diaz-MercadoJBE15}, the author introduces an approach for specifying robot tasks in formal language and generating controllers for a team of robots. 
However, the type of robot tasks is limited to non-reactive, i.e. the environment is assume static and the robot behavior does not depends on the environment state.
The framework in \cite{HKG2009} allows reactive robot task, such as "if the robot sense a soda can, bring the can to the kitchen".
However, the task specification only issues commands for a single robot.
In order to specify reactive tasks for a team of robots, we need to extend the specification language to able to express  multi-robot tasks.

\subsection{Tasks Allocation}
The framework employed in this work will generate a centralized controller for a team of robot.
We will then distribute tasks to each robots in a synchronized process.
One method is to handle task allocation in the high-level specification.
Therefore, task distribution for each robot will be encoded in the synthesized controller.
The drawback for this method is that, whenever a new task allocation is required,
possibly due to failing to find a feasible trajectory, the entire discrete controller needs to be resynthesized.
Another method for task allocation is to treat all robots as one virtual robot with multiple redundant action abilities.
The specification will only describe tasks for this virtual robot,
and the task allocation will be determined when executing the synthesized controller.
The disadvantage of this method is that the algorithm will spend more time during execution for computing the task allocation plan.
Both method will be implemented in this work.
We will compare the performance of both methods and choose the better one.

\subsection{Feedback from Trajectory Planner}
Once task are allocated to each robot, the task planner will invoke the trajectory planner to move each robot arm to desired location without collision.
In the situation where the trajectory planner fails to find a feasible path for an arm,
it will provide feedback to the task planner about such failure.
The task planner will then incorporate the information into the task specification and come up with a new discrete plan.
If no plan can be created, the task planner will notify the user the possible cause of failure, e.g. the robot arm cannot reach the desire position. 